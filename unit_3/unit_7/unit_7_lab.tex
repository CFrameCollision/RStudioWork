% Options for packages loaded elsewhere
\PassOptionsToPackage{unicode}{hyperref}
\PassOptionsToPackage{hyphens}{url}
%
\documentclass[
]{article}
\usepackage{amsmath,amssymb}
\usepackage{iftex}
\ifPDFTeX
  \usepackage[T1]{fontenc}
  \usepackage[utf8]{inputenc}
  \usepackage{textcomp} % provide euro and other symbols
\else % if luatex or xetex
  \usepackage{unicode-math} % this also loads fontspec
  \defaultfontfeatures{Scale=MatchLowercase}
  \defaultfontfeatures[\rmfamily]{Ligatures=TeX,Scale=1}
\fi
\usepackage{lmodern}
\ifPDFTeX\else
  % xetex/luatex font selection
\fi
% Use upquote if available, for straight quotes in verbatim environments
\IfFileExists{upquote.sty}{\usepackage{upquote}}{}
\IfFileExists{microtype.sty}{% use microtype if available
  \usepackage[]{microtype}
  \UseMicrotypeSet[protrusion]{basicmath} % disable protrusion for tt fonts
}{}
\makeatletter
\@ifundefined{KOMAClassName}{% if non-KOMA class
  \IfFileExists{parskip.sty}{%
    \usepackage{parskip}
  }{% else
    \setlength{\parindent}{0pt}
    \setlength{\parskip}{6pt plus 2pt minus 1pt}}
}{% if KOMA class
  \KOMAoptions{parskip=half}}
\makeatother
\usepackage{xcolor}
\usepackage[margin=1in]{geometry}
\usepackage{color}
\usepackage{fancyvrb}
\newcommand{\VerbBar}{|}
\newcommand{\VERB}{\Verb[commandchars=\\\{\}]}
\DefineVerbatimEnvironment{Highlighting}{Verbatim}{commandchars=\\\{\}}
% Add ',fontsize=\small' for more characters per line
\usepackage{framed}
\definecolor{shadecolor}{RGB}{248,248,248}
\newenvironment{Shaded}{\begin{snugshade}}{\end{snugshade}}
\newcommand{\AlertTok}[1]{\textcolor[rgb]{0.94,0.16,0.16}{#1}}
\newcommand{\AnnotationTok}[1]{\textcolor[rgb]{0.56,0.35,0.01}{\textbf{\textit{#1}}}}
\newcommand{\AttributeTok}[1]{\textcolor[rgb]{0.13,0.29,0.53}{#1}}
\newcommand{\BaseNTok}[1]{\textcolor[rgb]{0.00,0.00,0.81}{#1}}
\newcommand{\BuiltInTok}[1]{#1}
\newcommand{\CharTok}[1]{\textcolor[rgb]{0.31,0.60,0.02}{#1}}
\newcommand{\CommentTok}[1]{\textcolor[rgb]{0.56,0.35,0.01}{\textit{#1}}}
\newcommand{\CommentVarTok}[1]{\textcolor[rgb]{0.56,0.35,0.01}{\textbf{\textit{#1}}}}
\newcommand{\ConstantTok}[1]{\textcolor[rgb]{0.56,0.35,0.01}{#1}}
\newcommand{\ControlFlowTok}[1]{\textcolor[rgb]{0.13,0.29,0.53}{\textbf{#1}}}
\newcommand{\DataTypeTok}[1]{\textcolor[rgb]{0.13,0.29,0.53}{#1}}
\newcommand{\DecValTok}[1]{\textcolor[rgb]{0.00,0.00,0.81}{#1}}
\newcommand{\DocumentationTok}[1]{\textcolor[rgb]{0.56,0.35,0.01}{\textbf{\textit{#1}}}}
\newcommand{\ErrorTok}[1]{\textcolor[rgb]{0.64,0.00,0.00}{\textbf{#1}}}
\newcommand{\ExtensionTok}[1]{#1}
\newcommand{\FloatTok}[1]{\textcolor[rgb]{0.00,0.00,0.81}{#1}}
\newcommand{\FunctionTok}[1]{\textcolor[rgb]{0.13,0.29,0.53}{\textbf{#1}}}
\newcommand{\ImportTok}[1]{#1}
\newcommand{\InformationTok}[1]{\textcolor[rgb]{0.56,0.35,0.01}{\textbf{\textit{#1}}}}
\newcommand{\KeywordTok}[1]{\textcolor[rgb]{0.13,0.29,0.53}{\textbf{#1}}}
\newcommand{\NormalTok}[1]{#1}
\newcommand{\OperatorTok}[1]{\textcolor[rgb]{0.81,0.36,0.00}{\textbf{#1}}}
\newcommand{\OtherTok}[1]{\textcolor[rgb]{0.56,0.35,0.01}{#1}}
\newcommand{\PreprocessorTok}[1]{\textcolor[rgb]{0.56,0.35,0.01}{\textit{#1}}}
\newcommand{\RegionMarkerTok}[1]{#1}
\newcommand{\SpecialCharTok}[1]{\textcolor[rgb]{0.81,0.36,0.00}{\textbf{#1}}}
\newcommand{\SpecialStringTok}[1]{\textcolor[rgb]{0.31,0.60,0.02}{#1}}
\newcommand{\StringTok}[1]{\textcolor[rgb]{0.31,0.60,0.02}{#1}}
\newcommand{\VariableTok}[1]{\textcolor[rgb]{0.00,0.00,0.00}{#1}}
\newcommand{\VerbatimStringTok}[1]{\textcolor[rgb]{0.31,0.60,0.02}{#1}}
\newcommand{\WarningTok}[1]{\textcolor[rgb]{0.56,0.35,0.01}{\textbf{\textit{#1}}}}
\usepackage{graphicx}
\makeatletter
\def\maxwidth{\ifdim\Gin@nat@width>\linewidth\linewidth\else\Gin@nat@width\fi}
\def\maxheight{\ifdim\Gin@nat@height>\textheight\textheight\else\Gin@nat@height\fi}
\makeatother
% Scale images if necessary, so that they will not overflow the page
% margins by default, and it is still possible to overwrite the defaults
% using explicit options in \includegraphics[width, height, ...]{}
\setkeys{Gin}{width=\maxwidth,height=\maxheight,keepaspectratio}
% Set default figure placement to htbp
\makeatletter
\def\fps@figure{htbp}
\makeatother
\setlength{\emergencystretch}{3em} % prevent overfull lines
\providecommand{\tightlist}{%
  \setlength{\itemsep}{0pt}\setlength{\parskip}{0pt}}
\setcounter{secnumdepth}{-\maxdimen} % remove section numbering
\ifLuaTeX
  \usepackage{selnolig}  % disable illegal ligatures
\fi
\usepackage{bookmark}
\IfFileExists{xurl.sty}{\usepackage{xurl}}{} % add URL line breaks if available
\urlstyle{same}
\hypersetup{
  pdftitle={unit\_7\_lab.R},
  pdfauthor={byrds},
  hidelinks,
  pdfcreator={LaTeX via pandoc}}

\title{unit\_7\_lab.R}
\author{byrds}
\date{2024-11-07}

\begin{document}
\maketitle

\begin{Shaded}
\begin{Highlighting}[]
\FunctionTok{library}\NormalTok{(plyr)}
\FunctionTok{library}\NormalTok{(DescTools)}
\FunctionTok{library}\NormalTok{(ggplot2)}
\FunctionTok{library}\NormalTok{(dplyr)}
\end{Highlighting}
\end{Shaded}

\begin{verbatim}
## 
## Attaching package: 'dplyr'
\end{verbatim}

\begin{verbatim}
## The following objects are masked from 'package:plyr':
## 
##     arrange, count, desc, failwith, id, mutate, rename, summarise,
##     summarize
\end{verbatim}

\begin{verbatim}
## The following objects are masked from 'package:stats':
## 
##     filter, lag
\end{verbatim}

\begin{verbatim}
## The following objects are masked from 'package:base':
## 
##     intersect, setdiff, setequal, union
\end{verbatim}

\begin{Shaded}
\begin{Highlighting}[]
\FunctionTok{library}\NormalTok{(RColorBrewer)}
\FunctionTok{library}\NormalTok{(hrbrthemes)}
\FunctionTok{library}\NormalTok{(tidyverse)}
\end{Highlighting}
\end{Shaded}

\begin{verbatim}
## -- Attaching core tidyverse packages ------------------------ tidyverse 2.0.0 --
## v forcats   1.0.0     v stringr   1.5.1
## v lubridate 1.9.3     v tibble    3.2.1
## v purrr     1.0.2     v tidyr     1.3.1
## v readr     2.1.5
\end{verbatim}

\begin{verbatim}
## -- Conflicts ------------------------------------------ tidyverse_conflicts() --
## x dplyr::arrange()   masks plyr::arrange()
## x purrr::compact()   masks plyr::compact()
## x dplyr::count()     masks plyr::count()
## x dplyr::desc()      masks plyr::desc()
## x dplyr::failwith()  masks plyr::failwith()
## x dplyr::filter()    masks stats::filter()
## x dplyr::id()        masks plyr::id()
## x dplyr::lag()       masks stats::lag()
## x dplyr::mutate()    masks plyr::mutate()
## x dplyr::rename()    masks plyr::rename()
## x dplyr::summarise() masks plyr::summarise()
## x dplyr::summarize() masks plyr::summarize()
## i Use the conflicted package (<http://conflicted.r-lib.org/>) to force all conflicts to become errors
\end{verbatim}

\begin{Shaded}
\begin{Highlighting}[]
\FunctionTok{library}\NormalTok{(viridis)}
\end{Highlighting}
\end{Shaded}

\begin{verbatim}
## Loading required package: viridisLite
\end{verbatim}

\begin{Shaded}
\begin{Highlighting}[]
\FunctionTok{library}\NormalTok{(forcats)}
\FunctionTok{library}\NormalTok{(readxl)}

\FunctionTok{library}\NormalTok{(readxl)}
\NormalTok{Module\_4\_Lab\_Data }\OtherTok{\textless{}{-}} \FunctionTok{read\_excel}\NormalTok{(}\StringTok{"C:/Users/byrds/Downloads/Module 4 Lab Data.xlsx"}\NormalTok{, }
                                \AttributeTok{col\_types =} \FunctionTok{c}\NormalTok{(}\StringTok{"numeric"}\NormalTok{, }\StringTok{"numeric"}\NormalTok{))}

\NormalTok{data\_2017 }\OtherTok{\textless{}{-}} \FunctionTok{na.omit}\NormalTok{(Module\_4\_Lab\_Data[[}\DecValTok{1}\NormalTok{]])}
\NormalTok{data\_2007 }\OtherTok{\textless{}{-}}\NormalTok{ Module\_4\_Lab\_Data[[}\DecValTok{2}\NormalTok{]]}

\CommentTok{\# Stat sig of differences in age of consistent prayer}

\FunctionTok{t.test}\NormalTok{(data\_2007, data\_2017, }\AttributeTok{alternative =} \FunctionTok{c}\NormalTok{(}\StringTok{"two.sided"}\NormalTok{),}
       \AttributeTok{conf.level =} \FloatTok{0.95}\NormalTok{)}
\end{Highlighting}
\end{Shaded}

\begin{verbatim}
## 
##  Welch Two Sample t-test
## 
## data:  data_2007 and data_2017
## t = -6.1789, df = 1327.7, p-value = 8.572e-10
## alternative hypothesis: true difference in means is not equal to 0
## 95 percent confidence interval:
##  -7.102717 -3.679441
## sample estimates:
## mean of x mean of y 
##  52.29478  57.68586
\end{verbatim}

\begin{Shaded}
\begin{Highlighting}[]
\CommentTok{\# Tests for significance between data sets}

\CommentTok{\# Create a data frame of 21≥}
\NormalTok{lessthan22\_07 }\OtherTok{\textless{}{-}} \FunctionTok{subset}\NormalTok{(data\_2007, data\_2007 }\SpecialCharTok{\textless{}} \DecValTok{22}\NormalTok{)}
\NormalTok{lessthan22\_17 }\OtherTok{\textless{}{-}} \FunctionTok{subset}\NormalTok{(data\_2017, data\_2017 }\SpecialCharTok{\textless{}} \DecValTok{22}\NormalTok{)}

\CommentTok{\# Create a data frame of 69≤}
\NormalTok{greaterthan68\_07 }\OtherTok{\textless{}{-}} \FunctionTok{subset}\NormalTok{(data\_2007, data\_2007 }\SpecialCharTok{\textgreater{}} \DecValTok{68}\NormalTok{)}
\NormalTok{greaterthan68\_17 }\OtherTok{\textless{}{-}} \FunctionTok{subset}\NormalTok{(data\_2017, data\_2017 }\SpecialCharTok{\textgreater{}} \DecValTok{68}\NormalTok{)}

\CommentTok{\# Stat sig for 21{-}}
\FunctionTok{t.test}\NormalTok{(lessthan22\_07, lessthan22\_17, }\AttributeTok{alternative =} \FunctionTok{c}\NormalTok{(}\StringTok{"less"}\NormalTok{),}
       \AttributeTok{conf.levels =} \FloatTok{0.95}\NormalTok{)}
\end{Highlighting}
\end{Shaded}

\begin{verbatim}
## 
##  Welch Two Sample t-test
## 
## data:  lessthan22_07 and lessthan22_17
## t = 0.50175, df = 1.0571, p-value = 0.6499
## alternative hypothesis: true difference in means is less than 0
## 95 percent confidence interval:
##      -Inf 9.532605
## sample estimates:
## mean of x mean of y 
##  19.26316  18.50000
\end{verbatim}

\begin{Shaded}
\begin{Highlighting}[]
\FunctionTok{t.test}\NormalTok{(lessthan22\_07, lessthan22\_17, }\AttributeTok{alternative =} \FunctionTok{c}\NormalTok{(}\StringTok{"greater"}\NormalTok{),}
       \AttributeTok{conf.levels =} \FloatTok{0.95}\NormalTok{)}
\end{Highlighting}
\end{Shaded}

\begin{verbatim}
## 
##  Welch Two Sample t-test
## 
## data:  lessthan22_07 and lessthan22_17
## t = 0.50175, df = 1.0571, p-value = 0.3501
## alternative hypothesis: true difference in means is greater than 0
## 95 percent confidence interval:
##  -8.006289       Inf
## sample estimates:
## mean of x mean of y 
##  19.26316  18.50000
\end{verbatim}

\begin{Shaded}
\begin{Highlighting}[]
\CommentTok{\# Stat sig for 69+}
\FunctionTok{t.test}\NormalTok{(greaterthan68\_07, greaterthan68\_17, }\AttributeTok{alternative =} \FunctionTok{c}\NormalTok{(}\StringTok{"less"}\NormalTok{),}
       \AttributeTok{conf.levels =} \FloatTok{0.95}\NormalTok{)}
\end{Highlighting}
\end{Shaded}

\begin{verbatim}
## 
##  Welch Two Sample t-test
## 
## data:  greaterthan68_07 and greaterthan68_17
## t = 0.11525, df = 298.76, p-value = 0.5458
## alternative hypothesis: true difference in means is less than 0
## 95 percent confidence interval:
##      -Inf 1.226644
## sample estimates:
## mean of x mean of y 
##  76.61905  76.53896
\end{verbatim}

\begin{Shaded}
\begin{Highlighting}[]
\FunctionTok{t.test}\NormalTok{(greaterthan68\_07, greaterthan68\_17, }\AttributeTok{alternative =} \FunctionTok{c}\NormalTok{(}\StringTok{"greater"}\NormalTok{),}
       \AttributeTok{conf.levels =} \FloatTok{0.95}\NormalTok{)}
\end{Highlighting}
\end{Shaded}

\begin{verbatim}
## 
##  Welch Two Sample t-test
## 
## data:  greaterthan68_07 and greaterthan68_17
## t = 0.11525, df = 298.76, p-value = 0.4542
## alternative hypothesis: true difference in means is greater than 0
## 95 percent confidence interval:
##  -1.066471       Inf
## sample estimates:
## mean of x mean of y 
##  76.61905  76.53896
\end{verbatim}

\end{document}
